\chapter{Anhang}

Some guidelines and examples are given in the following.

\section{Figures}

A simple example of a figure can be found in \fig{fig:simple_figure}. A more complex figure including subfigures is show in \fig{fig:figure_with_subfigures}. Here each subfigure can be addressed separately (e.g., \fig{fig:subfigure1} and \fig{fig:subfigure2}). Please use vector graphics (pdf, eps obtained from svg, etc.) whenever possible. Pixel formats like jpeg, bmp, etc. should only be used for real photographs.

\begin{figure}[!h]
	\centering
	\fbox{\parbox{5cm}{\centering ~\vspace{1.5cm}\\Dummy\\~\vspace{1.5cm}}} %replace this line by: \includegraphics{path to image}
	\caption{Simple figure}
	\label{fig:simple_figure}
\end{figure}

\begin{figure}[!h]
	\centering
	\begin{subfigure}[b]{7cm}
		\centering
		\fbox{\parbox{5cm}{\centering ~\vspace{1.5cm}\\Dummy\\~\vspace{1.5cm}}} %replace this line by: \includegraphics{path to image}
		\caption{Caption of subfigure a (can be empty)}
		\label{fig:subfigure1}
	\end{subfigure}
	\begin{subfigure}[b]{7cm}
		\centering
		\fbox{\parbox{5cm}{\centering ~\vspace{1.5cm}\\Dummy\\~\vspace{1.5cm}}} %replace this line by: \includegraphics{path to image}
		\caption{Caption of subfigure b (can be empty)}
		\label{fig:subfigure2}
	\end{subfigure}
	\caption{Figure using subfigures}
	\label{fig:figure_with_subfigures}
\end{figure}


\section{Tables}

Examples of tables can be found in \tab{tab:simple_table} and \tab{tab:complex_table}. In general vertical lines are not necessary and should be avoided (see \cite{Fear05} for more about table styles).

\begin{table}[!h]
	\renewcommand{\arraystretch}{1.1}
	\caption{A very simple table}
	\label{tab:simple_table}
	\centering
	\begin{tabular}{cccc}
		\toprule
		& Apple & Orange & Banana \\
		\midrule
		Colour       & green & orange & yellow\\
		\bottomrule
	\end{tabular}
\end{table}

\begin{table}[!h]
	\renewcommand{\arraystretch}{1.1}
	\caption{An example of a more complex table}
	\label{tab:complex_table}
	\centering
	\begin{tabular}{ccC{1cm}C{1cm}C{1cm}C{1cm}C{1cm}C{1cm}C{1cm}C{1cm}C{1cm}C{1cm}C{1cm}}
		\toprule
		& & \multicolumn{4}{c}{RPAG algorithm} & \multicolumn{5}{c}{RPAGT (proposed)}\\
		\cmidrule(rl){3-6} \cmidrule(rl){7-11}
		$N$ & $N_\text{uq}$ & S & add ops & pure reg. & reg. ops & S & add ops & pure reg. & reg. ops & impr.\\
		\cmidrule(rl){1-11}
		6   & 3  & 3 & 8  & 1 & 9  & 2 & 5  & 0 & 5  & 44.4\% \\
		10  & 5  & 3 & 10 & 3 & 13 & 2 & 6  & 2 & 8  & 38.5\% \\
		13  & 7  & 3 & 14 & 2 & 16 & 2 & 8  & 2 & 10 & 37.5\% \\
		20  & 10 & 3 & 15 & 4 & 18 & 2 & 9  & 3 & 12 & 33.3\% \\
		28  & 14 & 3 & 20 & 3 & 23 & 2 & 15 & 2 & 17 & 26.1\% \\
		41  & 21 & 3 & 31 & 1 & 32 & 2 & 23 & 2 & 25 & 21.9\% \\
		61  & 31 & 3 & 39 & 3 & 42 & 2 & 32 & 2 & 34 & 19.0\% \\
		119 & 54 & 3 & 62 & 7 & 69 & 2 & 56 & 1 & 57 & 17.4\% \\
		151 & 71 & 3 & 79 & 4 & 83 & 2 & 72 & 2 & 74 & 10.8\% \\
		\cmidrule(rl){1-11}
		avg.: & 24 & & 30.89 & 3.56 & 33.89 & & 25.11 & 1.78 & 26.89 & 27.7\% \\
		\bottomrule
	\end{tabular}
\end{table}

\section{ToDo's}

During the writing of the thesis, ToDo's in the text can be highlighted using \verb|\todo|. Notes at the border of the text can be done using \verb|\todom|.
\todo{This has to be more extended}
\todom{ToDo remark at the border}
