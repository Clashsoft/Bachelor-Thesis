\chapter{Ergebnisse}\label{ch:ergebnisse}

Unter Berufung auf die in Kapitel~\ref{ch:goals} festgelegten Ziele soll nun eine Einschätzung erfolgen, inwiefern diese erreicht wurden.
Weiterhin gilt es zu betrachten, welche Schwierigkeiten bei der Umsetzung überwunden und welche Erkenntnisse dabei gewonnen wurden.

Das FulibScenarios-Projekt stellte aufgrund seines großen Umfangs eine Herausforderung dar.
Diese begann mit dem Design der Sprache, die gezielte Anpassung an den vorgesehenen Einsatz im Story Driven Modeling erforderte.
Der Einsatz natürlicher Sprache setzte neue Ideen hinsichtlich der Syntax voraus, da nicht auf Konstrukte wie geschweifte Klammern oder Einrückung zurückgegriffen werden konnte, die in anderen Programmiersprachen üblich sind.
Darüber hinaus ermöglicht die natürliche Sprache eine leichte Verständlichkeit für Leser ohne technischen Hintergrund, da Syntax und Semantik aus dem Alltag bekannt sind.
Das Schreiben der Scenarios setzt jedoch aufgrund der syntaktischen Anforderungen spezifische Fachkenntnisse voraus, die zunächst durch die an Softwareentwickler gerichtete Dokumentation und andere Materialien erlernt werden müssen.
Ohne Einschränkungen wären jedoch die eindeutige Semantik und somit die Ableitung von Diagrammen und Tests nicht möglich.
Auch die Integration in Markdown, die letztlich die Verwendung als Dokumentation erlaubt, musste in der Designphase beachtet werden.
Das Format ermöglicht den schnellen Einstieg in die Grundlagen der Scenario-Sprache, da Markdown vielen Entwicklern durch Plattformen wie GitHub oder StackOverflow bekannt ist.

Die Entwicklung des Compilers setzte eine gut durchdachte Architektur voraus, um die Vielzahl von Funktionalität übersichtlich und wartbar zu machen.
Der Gebrauch von FulibScenarios wurde von Studierenden zwar nur mittelmäßig (Durchschnittsnote 2,7) bewertet, dies ließ sich aber zu großen Teilen auf die mangelnde Dokumentation und andere, inzwischen behobene, Probleme mit Fehlermeldungen und Beispiel-Scenarios zurückführen.
Das FulibScenarios-Projekt ist noch nicht vollendet, denn weitere Änderungen und neue Features sind bereits geplant.
Beispiele dafür sind arithmetische Operatoren, Vererbung, erweiterte Textverarbeitung durch Templates und die Definition des Datenmodells ohne Angabe von Beispielobjekten.

Die Implementierung der Mustererkennung wurde durch die FulibTables-Bibliothek stark unterstützt.
Auch die Erweiterung, die das Abstrahieren über Namen erlaubte, konnte mit relativ geringem Aufwand durchgeführt werden.
Die Einbettung in die Scenario-Sprache stellte jedoch eine besondere Herausforderung dar.
Die Konzepte von FulibTables mussten dafür in ein umgängliches, natürliches Sprachformat übersetzt werden.
Wie das Beispiel im vorherigen Kapitel unter~\ref{subsec:assignment-pattern-matching} gezeigt hat, ist es dennoch gelungen, eine verständliche Syntax zu definieren und zu implementieren.
Auch hier ist es angebracht, neue Einsatzmöglichkeiten zu erproben und die Funktionalität zu erweitern.
Da es sich um neue Sprachkonzepte handelt, konnten diese noch nicht in einem praktischen Kontext angewendet werden.

Die Hauptseite von fulib.org hat sich seit ihrer Erstellung als nützliches Werkzeug bewährt.
Beim Verfassen dieser Arbeit wurde sie vermehrt eingesetzt, um Beispiele der Scenario-Sprache zu erstellen und zu überprüfen.
Ohne die Seite wäre das mühsame Einrichten eines Projekts vonnöten gewesen.
Der einfache Einstieg war letztlich auch in der Vorlesung "Programmieren und Modellieren" von Vorteil.
Wie sich aus der sehr guten Bewertung der bearbeiteten Hausaufgaben ableiten lässt, konnten die Studierenden durch die Beispiel-Scenarios und das sofortige Feedback schnell und effektiv die notwendigen Aspekte der Scenario-Sprache erlernen.
Besonders die Aufzeichnung von Anfragen ermöglichte es, Probleme in Sprache und Compiler zu erkennen und häufig auftretende Fehlermeldungen mit besseren Hinweisen auszustatten.
Die unter~\ref{subsec:students-view} analysierte Umfrage zeigte, welche Features in Zukunft priorisiert hinzugefügt werden sollten.

Bei der Entwicklung der Assignment-Funktionalität auf fulib.org konnte neben technischen Erkenntnissen in der Frontendentwicklung eine wertvolle Erfahrung gewonnen werden.
Da die Webseite ursprünglich nur mit JavaScript, \ac{html} und \ac{css} geschrieben war, wurde schnell deutlich, dass deren alleinige Verwendung nicht für die steigenden Anforderungen und die damit verbundene Komplexität geeignet sind.
Der Umstieg auf das Angular-Framework konnte diese Einschränkung umgehen.
Die in Kapitel~\ref{ch:goals} festgelegten Ziele für Aufgabenblätter konnten umgesetzt werden.
Vom Anlegen der Aufgabenblätter über die automatische Überprüfung bis hin zur Abgabe und Bewertung von Lösungen wurde der gesamte Prozess abgedeckt.
Kurse bieten eine moderne Alternative zu klassischen Hausaufgaben und festigen den Einsatz automatischer Korrektur, wodurch sie selbstständiges Lernen ermöglichen.
Auch fulib.org ist als Projekt noch nicht abgeschlossen.
Ein wichtiger Aspekt, der nicht im Rahmen dieser Arbeit lag, ist die Implementierung eines vollwertigen Benutzersystems.
Der Einsatz der Tokens zur Zugriffskontrolle soll dadurch ersetzt werden.
Mit einem Benutzersystem verbunden sind Bedenken zu Sicherheit und Datenschutz, die Aufgrund ihres Umfangs nicht näher betrachtet wurden.
Auch der Einsatz in einer größeren Testgruppe von Studierenden ist in Zukunft ratsam, um weitere Verbesserungsmöglichkeiten zu ermitteln und realisierbar zu machen.
