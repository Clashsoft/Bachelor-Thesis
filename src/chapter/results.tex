\chapter{Ergebnisse}\label{ch:ergebnisse}

Unter Berufung auf die in Kapitel~\ref{ch:motivation-and-goals} gesteckten Ziele soll nun eine Einschätzung erfolgen, inwiefern diese erreicht wurden.
Außerdem gilt es zu betrachten, \todo{welche Besonderen Schwierigkeiten überwunden und welche Erkenntnisse damit gewonnen wurden}.

Das FulibScenarios-Projekt stellte aufgrund des großen Umfangs eine Herausforderung dar.
Dies begann mit dem Design der Sprache, das gezielte Anpassung an den vorgesehenen Einsatz im Story Driven Modeling erforderte.
Der Einsatz natürlicher Sprache erforderte neue Ideen hinsichtlich der Syntax, da nicht auf in anderen Programmiersprachen übliche Konstrukte wie geschweifte Klammern oder Einrückung zurückgegriffen werden konnte.
Auch die Integration in Markdown, die letztlich die Verwendung als Dokumentation erlaubt, musste dabei beachtet werden.
Die Entwicklung des Compilers setzte eine bedachte Architektur voraus, um die Vielzahl von Funktionalität übersichtlich und wartbar zu machen.
Der Gebrauch von FulibScenarios wurde von Studierenden zwar nur mittelmäßig bewertet,
dies lag aber zu großen Teilen an der mangelnden Dokumentation und anderen Problemen, die inzwischen behoben wurden.
Das FulibScenarios-Projekt ist jedoch noch nicht vollendet;
weitere Änderungen und neue Features sind bereits geplant.

Die Implementierung der Mustererkennung wurde sehr durch die bestehende FulibTables-Bibliothek unterstützt.
Auch die Erweiterung, die das Abstrahieren über Namen erlaubte, konnte mit relativ geringem Aufwand durchgeführt werden.
Einzig die Einbettung in die Scenario-Sprache stellte eine besondere Herausforderung dar.
Die Konzepte von FulibTables mussten dafür in ein umgängliches, natürliches Sprachformat umgewandelt werden.
Wie das Beispiel im vorherigen Kapitel gezeigt hat, ist es dennoch gelungen, eine verständliche Syntax zu definieren und zu implementieren.
Auch hier ist es angebracht, neue Einsatzmöglichkeiten zu erproben und die Funktionalität zu erweitern.
Da es sich um neue Sprachkonzepte handelt, konnten diese insbesondere noch nicht praktisch angewendet werden.

Die Hauptseite von fulib.org hat sich seit ihrer Erstellung als nützliches Werkzeug bewährt.
Auch beim Verfassen dieser Arbeit wurde sie vermehrt eingesetzt, um Beispiel der Scenario-Sprache zu erstellen und zu prüfen.
Ohne die Seite wäre das mühsame Einrichten eines Projekts von Nöten gewesen.
Der einfache Einstieg war letztlich auch in der Vorlesung Programmieren und Modellieren von Vorteil.
Durch die Beispielszenarien und das sofortige Feedback konnten die Studierenden schnell die notwendigen Aspekte der Scenario-Sprache erlernen.
Besonders die Aufzeichnung von Anfragen ermöglichte es, Probleme in Sprache und Compiler zu erkennen und häufige Fehlerfälle mit besseren Hinweisen auszustatten.

Bei der Entwicklung der Assignment-Funktionalität auf fulib.org konnten viele Erkenntnisse in der Frontendentwicklung gewonnen werden.
Ursprünglich nur mit JavaScript, HTML und CSS geschrieben, wurde schnell deutlich, dass diese nicht für die steigenden Anforderungen und damit verbundene Komplexität geeignet sind.
Der Umstieg auf das Angular-Framework konnte dies umgänglich machen und gleichzeitig viele weitere Probleme beheben, die zuvor bestanden.
Die gesteckten Ziele bzgl.\ Aufgabenblättern konnten alle umgesetzt werden.
Letztlich wurde sogar das ursprünglich nicht geplante Feature der Kurse hinzugefügt, die besonders bei unbaufsichtigen Lehrangebot von Wert sind.
Aber auch fulib.org ist als Projekt noch nicht abgeschlossen.
Ein wichtiger Aspekt, der nicht im Rahmen dieser Arbeit lag, ist die Implementierung eines vollwertigen Benutzersystems.
Der Einsatz der Tokens zur Zugriffskontrolle soll damit ersetzt werden.
Mit einem Benutzersystem verbunden sind Bedenken zu Sicherheit und Datenschutz, die Aufgrund des Umfangs nicht näher betrachtet werden konnten.
