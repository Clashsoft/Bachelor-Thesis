\section{Sprache}\label{sec:language}

Kern dieser Arbeit ist eine neue Programmiersprache für textuelle Beispielszenarien.
Diese trägt den Namen Scenario-Sprache.
Ziel der Scenario-Sprache ist es, verständlich für jeden zu sein, der Englisch spricht.
Um als Programmiersprache funktionsfähig zu sein, hat sie im Gegensatz zu Englisch eine feste und eingeschränkte grammatikalische Struktur.
Somit handelt es sich um eine Untermenge der Englischen Sprache.
Des Weiteren basiert die Scenario-Sprache auf dem Markdown-Format,
das es erlaubt, einfachen Text mit Überschriften, Fett- und Kursivschreibung, Bildern u.ä.\ zu versehen.
Markdown-Dateien können leicht in HTML umgewandelt werden.
Somit können in der Scenario-Sprache verfasste Dateien als Dokumentation verwendet werden.
Im Folgenden werden einige Grundlagen der Scenario-Sprache dargestellt und erklärt.

\subsection{Grundlagen}\label{subsec:basics}

In der Scenario-Sprache verfasster Quellcode wird in \code{.md}-Dateien abgelegt.
Diese beginnen stets mit einer Überschrift, welche in Markdown mit dem \code{#}-Symbol beginnen.
Mit der Überschrift beginnt ein \emph{Scenario};
deren Text wird zu dessen Namen.
Eine Scenario-Datei kann mehrere Überschriften und damit mehrere Scenarios enthalten.

Nach einer Überschrift können ein oder mehrere Sätze und Unterüberschriften folgen;
diese Bilden den Rumpf des Scenarios.
Unterüberschriften sind an \code{##} am Anfang einer Zeile zu erkennen und ermöglichen die Strukturierung von langen Scenarios.

Im Rumpf von Scenarios gibt es einige Möglichkeiten, Kommentare zu hinterlassen.
Mit \code{//} wird wie in anderen Programmiersprachen ein Zeilenkommentar begonnen,
der mit dem nächsten Zeilenumbruch endet.
Dieser ist sowohl nach Umwaldeln des Markdown in HTML sichtbar,
als auch im erzeugten Java-Quellcode vorhanden.
Text der in runden Klammern \code{(...)} steht ist ebenfalls im HTML sichtbar,
jedoch nicht im Java-Quellcode.
Zuletzt können mit \code{<!-- ... -->} Kommentare eingebettet werden,
die weder im HTML sichtbar noch im Java-Code vorhanden sind.

\todo{
The paper~\cite{explain}.
}
