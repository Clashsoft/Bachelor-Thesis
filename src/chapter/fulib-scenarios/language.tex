\section{Sprache}\label{sec:language}

Kern dieser Arbeit ist eine neue Programmiersprache für textuelle Beispielszenarien.
Diese trägt den Namen Scenario-Sprache.
Ziel der Scenario-Sprache ist es, verständlich für jeden zu sein, der Englisch spricht.
Um als Programmiersprache funktionsfähig zu sein, hat sie im Gegensatz zu Englisch eine feste und eingeschränkte grammatikalische Struktur.
Somit handelt es sich um eine Untermenge der Englischen Sprache.
Des Weiteren basiert die Scenario-Sprache auf dem Markdown-Format,
das es erlaubt, einfachen Text mit Überschriften, Fett- und Kursivschreibung, Bildern u.ä.\ zu versehen.
Markdown-Dateien können leicht in HTML umgewandelt werden.
Somit können in der Scenario-Sprache verfasste Dateien als Dokumentation verwendet werden.
Im Folgenden werden einige Grundlagen der Scenario-Sprache dargestellt und erklärt.

\subsection{Grundlagen}\label{subsec:basics}

In der Scenario-Sprache verfasster Quellcode wird in \code{.md}-Dateien abgelegt.
Diese beginnen stets mit einer Überschrift, welche in Markdown mit dem \code{#}-Symbol beginnen.
Mit der Überschrift beginnt ein \emph{Scenario};
deren Text wird zu dessen Namen.
Eine Scenario-Datei kann mehrere Überschriften und damit mehrere Scenarios enthalten.

Nach einer Überschrift können ein oder mehrere Sätze und Unterüberschriften folgen;
diese Bilden den Rumpf des Scenarios.
Unterüberschriften sind an \code{##} am Anfang einer Zeile zu erkennen und ermöglichen die Strukturierung von langen Scenarios.

Im Rumpf von Scenarios gibt es einige Möglichkeiten, Kommentare zu hinterlassen.
Mit \code{//} wird wie in anderen Programmiersprachen ein Zeilenkommentar begonnen,
der mit dem nächsten Zeilenumbruch endet.
Dieser ist sowohl nach Umwaldeln des Markdown in HTML sichtbar,
als auch im erzeugten Java-Quellcode vorhanden.
Text der in runden Klammern \code{(...)} steht ist ebenfalls im HTML sichtbar,
jedoch nicht im Java-Quellcode.
Zuletzt können mit \code{<!-- ... -->} Kommentare eingebettet werden,
die weder im HTML sichtbar noch im Java-Code vorhanden sind.

\subsection{Einfache Sätze und Ausdrücke}\label{subsec:simple-sentences-and-expressions}

Sätze bilden den Inhalt eines Scenarios und definieren dessen Ablauf.
Die Scenario-Sprache definiert eine Vielzahl von Sätzen,
die sich sowohl in ihrer Funktionalität ergänzen,
als auch syntaktische Alternativen mit gleicher Semantik füreinander darstellen.

Die einfachste Art von Satz ist der \code{Is}-Satz.
Dieser ermöglicht es, ein Objekt zu definieren und diesem Name und Typ zuzuweisen.
Ein Beispiel dafür ist

\code{Kassel is a City.}

Dabei ist \code{Kassel} der Name des Objekts,
und \code{City} dessen Typ.
\code{is} und \code{a} sind Schlüsselwörter.
Die äquivalente Java-Anweisung ist

\code{City kassel = new City();}

Zu beachten ist hier, dass die Klasse \code{City} nicht vorher deklariert werden muss.
Durch Verwenden des Namens wird diese automatisch angelegt.
Es gibt in der Scenario-Sprache keine Syntax für das manuelle Definieren von Klassen.

Ein weiterer einfacher Satz ist der \code{Has}-Satz.
Damit können Attribute von bereits bestehenden Objekten einen Wert zugewiesen bekommen.
Ein Beispiel dafür ist folgendes:

\code{Kassel has postcode 34117.}

Dabei ist \code{Kassel} der Name des Zielobjekts, in diesem Fall jenes, welches zuvor mit dem \code{Is}-Satz angelegt wurde.
\code{postcode} ist der Name des Attributs; \code{34117} der zuzuweisende Wert.
Aus diesem Satz wird der folgende Java-Code:

\code{kassel.setPostcode(34117)}

Wieder ist zu beachten, dass das Attribute \code{postcode} bzw.\ der Setter \code{setPostcode} nicht im Vorhinein deklariert wurde;
dieses wurde durch die Verwendung automatisch angelegt.
Durch den Ausdruck \code{34117} konnte ermittelt werden, dass der Typ dieses Attributes \code{int} sein muss.
Wäre der Wert stattdessen \code{D-34117}, was von der Scenario-Sprache als Zeichenkette verstanden wird,
hätte das Attribut den Typ \code{String} erhalten und der entsprechende Java-Code wäre \code{kassel.setPostcode("D-34117")}.

Da das Definieren von Objekten und die Zuweisung von Attributen sehr häufig in Kombination geschieht, bietet die Scenario-Sprache eine alternativ Satzart an, die beides gleichzeitig durchführt.
Diese Sätze heißen \code{There}-Sätze.
Die obigen Beispiele lassen sich mit einem There-Satz verkürzen:

\code{There is the City Kassel with postcode 34117.}

Hierbei sind die Schlüsselwörter \code{There is the} der Ersatz für \code{is},
während \code{has} durch \code{with} ersetzt wurde.
Der Java-Code ist äquivalent zu den beiden zuvor gezeigten Zeilen.

Bei erneuter Betrachtung des Java-Codes fällt auf, dass das Wort \code{Kassel} darin nur als Variablenname, jedoch nicht als Wert vorkommt.
Somit ist der Name der Stadt zur Laufzeit nicht zu ermitteln;
dies ist bei der Modellierung von Objektstrukturen i.d.R.\ unpraktisch.
Aus diesem Grund bieten \code{There}-Sätze eine Möglichkeit, eine Zeichenkette sowohl als Attributwert als auch als Variablennamen zu verwenden:

\code{There is a City with name Kassel and with postcode 34117.}

Hier ist zu sehen, dass in einem \code{There}-Satz mehrere Attributzuweisungen mit \code{with} möglich sind, indem sie durch \code{and} getrennt werden\footnote{
Alternativ können diese auch durch \code{,} (Komma) und \code{, and} (And mit Oxford-Komma) getrennt werden, da dies bei mehr als zwei \code{with} die Lesbarkeit erhöht.
}.
Das ansonsten unbenannte Objekt bezieht seinen Namen aus der ersten Attributzuweisung,
heißt also wieder \code{kassel}.
Der entsprechende Java-Code ist wie folgt:

\code{City kassel = new City();}\\
\code{kassel.setName("Kassel");}\\
\code{kassel.setPostcode(34117);}

\todo{
The paper~\cite{explain}.
}
