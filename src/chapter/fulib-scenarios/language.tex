\section{Sprache}\label{sec:language}

Kern dieser Arbeit ist eine neue Programmiersprache für textuelle Beispielszenarien.
Diese trägt den Namen Scenario-Sprache.
Ziel der Scenario-Sprache ist es, verständlich für jeden zu sein, der Englisch spricht.
Um als Programmiersprache funktionsfähig zu sein, hat sie im Gegensatz zu Englisch eine feste und eingeschränkte grammatikalische Struktur.
Somit handelt es sich um eine Untermenge der Englischen Sprache.
Des Weiteren basiert die Scenario-Sprache auf dem Markdown-Format,
das es erlaubt, einfachen Text mit Überschriften, Fett- und Kursivschreibung, Bildern u.ä.\ zu versehen.
Markdown-Dateien können leicht in HTML umgewandelt werden.
Somit können in der Scenario-Sprache verfasste Dateien als Dokumentation verwendet werden.
Im Folgenden werden einige Grundlagen der Scenario-Sprache dargestellt und erklärt.

\todo{
The paper~\cite{explain}.
}
