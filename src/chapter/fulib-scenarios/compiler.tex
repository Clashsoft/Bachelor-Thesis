\section{Compiler}\label{sec:compiler}

Damit eine Programmiersprache verwendbar ist, muss sie nicht nur eine Spezifikation haben, sondern auch ausführbar sein.
Dies kann durch direktes Ausführen des Programmtexts geschehen;
in diesem Fall wird das ausführende Programm Interpreter genannt.
FulibScenarios verwendet jedoch den zweiten Ansatz der Kompilierung.
Dabei werden die vorliegenden Quelltexte zunächst in eine andere Text- oder Binärsprache übersetzt.
Ziel der Kompilierung von FulibScenarios sind Java-Quelltextdateien.
Die Übersetzung von Markdown-Scenario-Quelltexte in Java-Quelltexte wird durch den Scenario-Compiler durchgeführt.
Dessen Architektur und Implementierung sind Inhalt dieses Abschnitts.
Dabei wird des öfteren auf das Beispiel in Listing~\ref{lst:CompilationExample.md} Bezug genommen, dessen Übersetzung in den einzelnen Teilabschnitten stückweise erarbeitet wird:

% TODO markdown
\codelisting{text}{chapter/fulib-scenarios/scenarios}{CompilationExample.md}{Beispiel-Szenario für diesen Abschnitt}

\subsection{Architektur}\label{subsec:compiler-architecture}

\todo{
Visitor Pattern~\cite{gof-design-patterns}.
Dragon Book~\cite{dragonbook}.
}

\begin{figure}
	\includegraphics[width=\textwidth]{chapter/fulib-scenarios/img/architecture.pdf}
	\caption{Compiler-Architektur}
	\label{fig:compiler-architecture}
\end{figure}

\subsection{Frontend - ANTLR v4}\label{subsec:frontend-antlr4}

Die Übersetzung der Markdown-Datei beginnt mit deren Einlesen und Umwandeln in verarbeitbare Daten.
Im Compilerbau wird die klassisch in zwei Phasen unterteilt:
Dem Lexer, der die Zeichenfolge in eine Liste von Wörtern mit Typinformationen, genannt Tokens, umwandelt;
und dem Parser, der die flache Liste von Tokens nach den Regeln der Grammatik in eine Baumstruktur bringt,
die Concrete Syntax Tree (CST) genannt wird.

Da die Umwandlung in Tokens und deren Strukturierung Anwendung in den meisten Compilern findet
und deren Implementierung immer nach einem festen Muster stattfindet,
existieren Tools die diesen Prozess vereinfachen.
Diese werden Compiler-Compiler, Lexer- und Parsergeneratoren genannt.

Das Frontend des Scenario-Compilers basiert auf dem ANTLR4-Parsergenerator~\cite{antlr4-reference}.
Mit diesem können Grammatiken in einem EBNF-ähnlichen Format spezifiziert werden.
Aus der Grammatik generiert das Tool dann Java-Code, der Dateien einlesen kann und einen CST generiert.

Zunächst soll betrachtet werden, wie die FulibScenarios-Grammatik definiert ist.
Dies beginnt mit der Definition des Lexers.
In Listing~\ref{lst:ScenarioLexer.g4} ist ein Ausschnitt der ANTLR4-Grammatik zu sehen, die diesen definiert.
Der Ausschnitt ist gerade ausreichend um das Scenario aus Listing~\ref{lst:CompilationExample.md} zu akzeptieren.

\codelisting{antlr}{chapter/fulib-scenarios/grammars}{ScenarioLexer.g4}{Ausschnitt der Scenario-Lexer-Grammatik}

Die Grammatik besteht aus mehreren Regeln, die einem Muster einen Namen zuordnen.
Dieser Name wird später den Tokens zugeordnet.
So haben Tokens mit dem Text \code{and} später den Namen \code{AND};
und jene, die eine Zahl darstellen, den namen \code{INTEGER}.
Im Folgenden werden Tokens mit der Kurzschreibweise \code{<name>(<text>)} bezeichnet;
beispielsweise \code{AND(and)} und \code{INTEGER(12)}.

Die rechte Seite jeder Regel ähnelt einem regulären Ausdruck.
Bei Schlüsselwörtern wie \code{a}, \code{is} und \code{with} muss der Text exakt entsprechen;
\code{There} darf auch kleingeschrieben werden.
Ein \code{HEADLINE}-Token entsteht, wenn auf ein \code{#}-Zeichen beliebig viele Zeichen und ein Zeilenumbruch folgen\footnote{
Dabei bedeutet \code{~[\n]*?} \outquote{beliebig viele Zeichen ausgenommen Zeilenumbruch}.}.
Ganze Zahlen können mit einem Minus-Zeichen beginnen und bestehen aus mindestens einer Ziffer.
Wörter beginnen mit einem Buchstaben, gefolgt von beliebig vielen Buchstaben, Ziffern, Apostrophen, Unterstrichen und Bindestrichen.
Die Regel \code{WS} sorgt durch die Angabe \code{-> skip} dafür, dass keine Whitespace-Zeichen zu Token werden.
Falls mehrere Regeln infrage kommen würden, wird zunächst die längstmögliche Übereinstimmung angewandt;
falls das nicht eindeutig bestimmbar ist, gewinnt die als erstes definierte Regel.
So wird aus dem Text \code{ampersand} nicht die Token-Folge \code{A(a), WORD(mpers), AND(and)}, sondern \code{WORD(ampersand)}.

Wendet man die Regeln aus Listing~\ref{lst:ScenarioLexer.g4} auf das Scenario aus Listing~\ref{lst:CompilationExample.md} an,
so erhält man die in Listing~\ref{lst:CompilationExampleTokens.txt} gezeigte Liste von Tokens.

\codelisting{text}{chapter/fulib-scenarios/grammars}{CompilationExampleTokens.txt}{Aus Listing~\ref{lst:CompilationExample.md} abgeleitete Token-Liste}

\todo{
Warum ANTLR4 sehr gut für die Sprachstruktur geeignet ist~\cite{adaptive-ll-star,antlr4-reference}.
}

\subsection{AST, Analyse und Transformation}\label{subsec:data-model-gentreesrc}

\todo{
AST mit GenTreeSrc;
eigenes Projekt, kurze Erklärung.
Prä-Transformation: Gruppieren von Sätzen zu Methoden
Analyse: Variablen, Konflikte bei Attributen und Assoziationen, Programmstruktur.
Transformation: Bauen von Klassenmodell, Vereinfachung des AST in Vorbereitung auf CodeGen.
}

\subsection{Codegenerierung - Fulib}\label{subsec:codegen-fulib}

\todo{
Modellgenerierung mit Fulib.
Testgenerierung selbst.
}
