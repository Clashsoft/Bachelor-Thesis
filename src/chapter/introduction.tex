\chapter{Einleitung}\label{ch:introduction}

Seit der Erfindung digitaler Computer in den 1940er-Jahren gibt es auch Software.
Durch deren zunehmenden Einsatz und größer werdende Anforderungen wurde diese mit der Zeit immer komplexer.
Mit dem Ziel, diese steigende Komplexität für einzelne Entwickler und ganze Teams überschaubar zu machen, ist die Fachrichtung der Softwaretechnik entstanden.
Sie befasst sich jedoch nicht nur mit der Entwicklung von Software im Sinne der Implementierung;
auch Erarbeiten von Anforderungen, Design, Dokumentation, und Qualitätssicherung gehören u.a.\ zu den Schwerpunkten der Fachrichtung.

Speziell für die genannten Aspekte ist das Verfahren des \emph{Story Driven Modeling}~\cite{sdm} entstanden.
Dessen Grundidee ist die Ausarbeitung von Beispielszenarien, sogenannten Stories, welche Anforderungen in Form von konkreten Abläufen spezifizieren.
Die Stories beschreiben konkrete Personen oder Dinge sowie Aktionen, die von ihnen durchgeführt werden.
Dabei werden sowohl Startsituation als auch Ergebnis angegeben.
Ziel der Stories ist es, für Domänenexperten ohne technischen Hintergrund verständlich zu sein.
In Stories ist insbesondere der Einsatz von konkreten Beispielen von Relevanz;
abstrakte und allgemeine Definitionen sind darin nicht geeignet.
Die Beispiele dokumentieren dabei die Verwendung der Software aus Sicht des Benutzers.
Die Stories ermöglichen ferner dem Entwickler die Festlegung eines Datenmodells, welches als Grundlage für die Implementierung dient.
Anhand der konkreten Beispiele lassen sich Objektdiagramme für jede Story erstellen.
Aus den Objektdiagrammen kann daraufhin ein Klassendiagramm abgeleitet werden.
Erst mit dem Klassendiagramm kann mit der eigentlichen Programmierung begonnen werden, indem zugehöriger Code geschrieben wird.
Ist das Datenmodell als Code verfasst, kann die Logik der Software implementiert werden.
Diese richtet sich nach den Abläufen der Stories, welche zunächst verallgemeinert werden müssen.
Nach Implementierung der Logik muss geprüft werden, ob dieser Vorgang korrekt durchgeführt wurde.
Für die Prüfung haben sich automatische Tests etabliert.
Die Stories bieten sich für diese als Grundlage an, indem deren Abläufe in einem Test nachprogrammiert werden.
Der Test soll sicherstellen, dass die Story gültig ist und auch bleibt, nachdem Änderungen an der Software durchgeführt wurden.

Das Befolgen des Story-Driven-Modeling-Ansatzes scheint zunächst etwas aufwendig.
Für jedes Feature müssen neben der eigentlichen Implementierung mindestens eine Story geschrieben, Objekt- und evtl.\ Klassendiagramme gezeichnet und Tests programmiert werden.
Bei Änderung der Anforderung müssen diese Aspekte angepasst und aktualisiert werden.
Diese Arbeit beschäftigt sich zunächst mit der Suche nach einer Lösung für dieses Aufwands-\ und Wartungsproblem.
Sie führt eine Möglichkeit ein, Diagramme, Tests und Logik automatisch aus Beispielszenarien abzuleiten.
Kapitel~\ref{ch:fulib-scenarios} befasst sich näher mit dieser Thematik.

Bei der Softwareentwicklung spielt auch die Lehre eine Rolle.
Die Modellierung zählt dabei zu den ersten Konzepten, die Studierende der Informatik in diesem Fachgebiet vermittelt bekommen.
Oft bestehen die Aufgaben der Lernenden daraus, selbstständig Beispielszenarien und Datenmodelle zu erarbeiten.
Dabei ist es einerseits schwierig, konkrete Vorgaben an das Datenmodell zu stellen, da dies der Aufgabenstellung den kreativen Aspekt und einen wesentlichen Eigenanteil beraubt.
Die fehlende Festlegung von Aufbau oder Klassen-, Attribut- und Assoziationsnamen erschwert andererseits das Prüfen von Lösungen auf Korrektheit.
Diese Arbeit beschäftigt sich mit diesem Dilemma, indem sie eine Form der Mustererkennung für Datenmodelle einführt, die über Benennung abstrahieren kann.
Kapitel~\ref{ch:pattern-matching} ist dieser speziellen Mustererkennung gewidmet.
Dabei wird auch erklärt, wie diese in das in Kapitel~\ref{ch:fulib-scenarios} geschaffene Werkzeug integriert wurde.
Zuletzt gilt es, beides in der Lehre einsetzbar und zugänglich zu machen.
Dazu wird in Kapitel~\ref{ch:fulib.org} eine Online-Plattform vorgestellt, die einen Anwendungsfall für die Werkzeuge schafft.
Welche Anforderungen diese erfüllen sollen ist Inhalt des nächsten Kapitels.
