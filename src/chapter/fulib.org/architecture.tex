\section{Architektur}\label{sec:architecture}

fulib.org verwendet eine einfache Architektur bestehend aus Frontend, Backend und Datenbank.
Da es sich um eine Webanwendung handelt, wird das Frontend im Browser ausgeführt.
Öffnet man die Seite, werden die benötigten HTML-, JavaScript- und CSS-Dateien vom Backend heruntergeladen und im Browser angezeigt.
Aus Sicht des Backends sind dies statische Ressourcen.
Sämtlicher Datenaustausch in der Webanwendung geschieht dann über REST/HTTP-Anfragen an das Backend.
Dabei wird JSON als Datenformat verwendet.
Für die persistente Datenspeicherung von Anfragelogs, Aufgabenblättern, Lösungen, Kursen und Kommentaren kommt eine Mongo-Datenbank zum Einsatz.

\subsection{Frontend}\label{subsec:frontend}

Das Frontend von fulib.org ist mit dem Angular~\cite{angular}-Framework implementiert.
Dieses gibt insbesondere eine Architektur vor, die Business Logic (Services) von UI-Logik (Components) trennt.
Services sind beispielsweise dafür zuständig, mit dem Backend Daten auszutauschen,
während Komponenten sich mit deren Darstellung befassen.
Letztere haben weiterhin als wiederverwendbare Elemente Bedeutung.
Sie ermöglichen es, gemeinsame Funktionalität zu verkapseln und an mehreren Orten zu verwenden.
Im Kontext der Assignments ist beispielsweise die Task-Liste eine Komponente, die auf mehreren Seiten zum Einsatz kommt.
Da Komponenten nur einmal implementiert und beliebig wiederverwendet werden können,
kann die Oberfläche konsistent und fehlerfrei gehalten werden.

Das Aussehen der Oberfläche von fulib.org basiert auf Bootstrap~4~\cite{bootstrap}.
Das CSS-Framework gibt vielen HTML-Elementen wie Buttons oder Eingabefeldern ein modernes Aussehen.
Weiterhin gibt ermöglicht es die einfache Anordnung von Elementen, die sich an verschiedene Bildschirmgrößen anpassen kann.
So ist fulib.org ohne großen Entwicklungsaufwand auch auf Smartphones o.ä.\ übersichtlich.
Bei dem im Footer einstellbaren Nachtmodus (Darkmode) handelt es sich um die Bootstrap-Darkmode~\cite{bootstrap-darkmode}.
Diese Eigenentwicklung passt die Farbgebung von Bootstrap so an, dass alle normalerweise weißen oder hellen Elemente in schwarz- bzw.\ grautönen erscheinen.
Dies schont bei dunkleren Lichtbedingung wie etwa abends das Auge.

Die Editorfenster für Scenarios und Java-Code sind mit der CodeMirror~\cite{codemirror}-Bibliothek implementiert.
Diese bietet einen Editor, der Syntaxhighlighting für viele Programmiersprachen unterstützt.
Mit Themes kann die Farbgebung angepasst werden, was auf fulib.org beim Wechsel in den Nachtmodus zu beobachten ist.
Zudem unterstützt der Editor eine Reihe von Addons, die zusätzliche Funktionalität einbringen können.
Beispielsweise können mit dem Lint-Addon Fehlermeldungen im Editor hervorgehoben werden.
Es ist geplant, dies im Scenario-Editor einzusetzen.

\subsection{Backend}\label{subsec:backend}

\todo{
Java Spark.
No Spring.
}

\subsection{Datenbank}\label{subsec:database}

\todo{
MongoDB\@.
}
