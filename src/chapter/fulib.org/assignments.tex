\section{Assignments}\label{sec:assignments}

Assignments bezeichnen ein Feature von fulib.org, welches sich mit dem Anlegen, Lösen und Bewerten von Aufgabenblättern sowie Online-Kursen beschäftigt.
Es handelt sich dabei um Modellierungsaufgaben in der Scenario-Sprache.
Diese können sowohl automatisch geprüft als auch manuell bewertet werden.
In diesem Kapitel wird der chronologische Ablauf beschrieben, der bei Aufgabenblättern durchlaufen wird.
Der Unterabschnitt~\ref{subsec:creation} befasst sich zunächst mit deren Anlegen aus Sicht des Kursleiters.
Daraufhin wird unter~\ref{subsec:solution} der Lösungsvorgang aus Perspektive der Studierenden erläutert.
In Unterabschnitt~\ref{subsec:correcting} wird als letzten Schritt das Vorgehen der Korrekteure zum Bewerten der Lösungen beschrieben.
Weiterführend wird die Kurs-Funktion unter~\ref{subsec:courses} vorgestellt, welche die Gruppierung von Aufgabenblättern sowie das selbstständige und unbeaufsichtigte Lernen in größeren Themengebieten ermöglichen.
Zuletzt widmet sich Unterabschnitt~\ref{subsec:assignment-pattern-matching} einem vollständigem Beispielassignment, welches die im vorherigen Kapitel eingeführte Mustererkennung einsetzt.

\subsection{Anlegen}\label{subsec:creation}

Das Anlegen von Assignments ist mit dem in Abbildung~\ref{fig:create-assignment} gezeigten Formular möglich.
Dafür muss lediglich im stets sichtbaren Seitenfooter unter ``Assignments'' der Menüeintrag ``Create Assignment'' ausgewählt werden.

\begin{figure}
    \centering
    \includegraphics[width=\textwidth]{chapter/fulib.org/img/create-assignment.png}
    \caption{Formular zum Anlegen von Assignments}
    \label{fig:create-assignment}
\end{figure}

Hier werden zunächst der Titel des Assignments sowie Name und E-Mail-Adresse des Kursleiters eingetragen.
Ebenso kann eine Deadline festgelegt werden.
Das mit ``Description'' bezeichnete Feld ist für die Aufgabenstellung vorgesehen.
Dabei wird Markdown-Syntax unterstützt; es können also Überschriften, Tabellen, Listen, Bilder etc.\ eingebracht werden.

Als nächstes werden die Teilaufgaben (Tasks) eingetragen.
Mit dem Button ``Add Task'' kann der Liste ein neuer Task hinzugefügt werden.
Jeder Task hat eine kurze Beschreibung und eine maximal erreichbare Punktzahl.
Der mit ``Verification'' bezeichnete Editor ist dafür vorgesehen, einen Teil eines Scenarios mit Expect-Sätzen einzutragen.
Ein Task wird später wie folgt automatisch bewertet:
Zunächst wird das unter ``Verification'' eingetragene Teilscenario an die Lösung des Studierenden angehängt.
Das entstehende Scenario wird dann kompiliert und die entstandenen Tests werden ausgeführt.
Erzeugt dies einen Compilerfehler und schlägt der Test fehl, gilt die Teilaufgabe als nicht bestanden und wird mit null Punkten bewertet.
Andernfalls wird die maximal erreichbare Punktzahl vergeben.

Dies lässt sich an einem einfachen Beispiel demonstrieren.
Angenommen, die Beschreibung eines Tasks lautet, dass ein Objekt \code{alice} mit dem Wert \code{Alice} in einem Attribut \code{name} erstellt werden soll.
Der folgende Verifikationscode könnte dies prüfen:

\begin{mdcodeblock}
    We expect that alice has name 'Alice'.
\end{mdcodeblock}

Eine mögliche Lösung wäre dann folgender Satz:

\begin{mdcodeblock}
    There is a Student with name Alice.
\end{mdcodeblock}

Bei deren Prüfung wird diese automatisch mit dem Verifizierungscode in ein vollständiges Scenario verpackt:

\begin{mdcodeblock}
    # Scenario
    There is a Student with name Alice.
    ## Verification
    We expect that alice has name 'Alice'.
\end{mdcodeblock}

Wird dieses kompiliert und der entstehende JUnit-Test ausgeführt, ist dieser erfolgreich.
Folglich wird auf den Task die volle Punktzahl vergeben.
Andererseits entspricht die folgende Lösung nicht der Aufgabenstellung:

\begin{mdcodeblock}
    There is a Student with name Bob.
\end{mdcodeblock}

Fügt man die Lösung wie oben mit dem Verifizierungscode zusammen und kompiliert das Resultat, wird der Scenario-Compiler aufgrund der fehlenden Variable \code{alice} eine Fehlermeldung bei \code{alice has name ...} erzeugen.
Die Fehlermeldung wird als Nichterfüllung des Tasks aufgefasst, wodurch er mit null Punkten bewertet wird.

Zur Übersichtlichkeit lassen sich die einzelnen Tasks mit dem ``+''- bzw.\ ``-''-Button ein- und ausklappen.
Mit der Schaltfläche rechts davon lassen sich Tasks durch Ziehen mit Maus bzw.\ Finger anordnen.
Der rote ``$\times$''-Button löscht den Task aus der Liste.

Unter der Task-Liste befinden sich die Eingabefenster für die Lösungsvorlage (Template Solution) und die Musterlösung (Sample Solution).
Beide lassen sich mit den entsprechenden Buttons ein- und ausklappen, um das Formular übersichtlicher zu machen.
Gibt man eine Vorlage an, so wird diese den Studierenden als Lösung vorgegeben, wenn sie das Assignment öffnen.
Die Aufgabenstellung könnte dann beispielsweise sein, dass das vorgegebene Scenario angepasst oder vervollständigt wird.
Dafür könnten mit ``\code{...}'' Lücken in der Vorlage gelassen werden, welche die Studierenden ausfüllen sollen.
Für das zuvor genannte Beispiel wäre beispielsweise \code{There is a ...} als Vorgabe möglich.

Die Musterlösung ist zwar optional, sie erlaubt jedoch dem Ersteller, seine Aufgabenstellung auf Lösbarkeit zu prüfen.
Dadurch können Fehler in der Aufgabenstellung oder der Verifizierung frühzeitig erkannt werden.
Bei Änderung der Musterlösung wird sie automatisch geprüft.
In der Task-Liste werden dann diejenigen Tasks rot markiert, deren Verifizierung fehlgeschlagen ist.
Dies deutet an, dass es entweder ein Problem in der Musterlösung oder im Verifizierungscode eines Tasks gibt.

Sämtliche Änderungen am Formular werden sofort im Browser gespeichert.
Dadurch wird Datenverlust beim Verlassen der Seite oder bei Ausfall der Internetverbindung vermieden.
Dennoch gibt es die Möglichkeit, den Entwurf des Assignments als Datei auf der Festplatte zu speichern.
Dies ist mit dem ``Export''-Button möglich.
Später kann die Datei dann wieder in das Formular übernommen werden, indem sie im Feld neben dem ``Import''-Button ausgewählt und anschließend auf diesen geklickt wird.
Die Import/Export-Funktion erlaubt ferner, ein Assignment für die spätere Bearbeitung zu speichern und ein neues zu erstellen.
Heruntergeladene Dateien können weiterhin über E-Mail oder Dateifreigabe an andere weitergegeben werden, die das Assignment dann importieren und prüfen oder verändern können.

Um ein fertiges Assignment zu veröffentlichen, genügt ein Klick auf den ``Submit''-Button.
Dieser bewirkt, dass das ausgefüllte Formular an den Server gesendet wird, der das erstellte Assignment speichert und zwei Links generiert.
Abbildung~\ref{fig:create-assignment-success} zeigt das Fenster, das sich daraufhin öffnet und beide Links enthält.

\begin{figure}
    \centering
    \includegraphics[width=\textwidth]{chapter/fulib.org/img/create-assignment-success.png}
    \caption{Fenster nach Anlegen eines Assignments}
    \label{fig:create-assignment-success}
\end{figure}

Der erste Link ist dafür vorgesehen, ihn an die Studierenden weiterzuleiten.
Öffnen sie diesen, zeigt sich die Seite, in der ihre Lösung des Assignments erstellt und abgegeben werden kann.
Dieser Vorgang ist Thema des nächsten Unterabschnitts.
Der zweite Link ist für den Kursleiter und die Korrekteure, da darunter eine Liste aller abgegebenen Lösungen zu finden ist.
Um darauf zugreifen zu können, wird das gezeigte Token benötigt.
Dieses muss vom Kursleiter an die Korrekteure weitergeleitet werden.
Ohne das Token-System könnten Studierende die Lösungen von Anderen einsehen und übernehmen.

Assignments sind nach dem Einreichen nicht mehr veränderbar.
Dies ist gewünscht, da das nachträgliche Verändern von Assignments, die schon von Studierenden bearbeitet wurden, für diese nachteilhaft sein kann.
Wird im Nachhinein ein Fehler in der Aufgabenstellung erkannt, kann das gleiche Assignment erneut eingereicht werden.
Dabei werden neue Links und Token generiert, die wieder entsprechend geteilt werden müssen.

Wählt man im Footer Assignments > My Assignments aus, gelangt man zur Übersicht der eigenen Assignments.
Abbildung~\ref{fig:my-assignments} zeigt, wie diese Seite aussehen kann.
Zu eigenen Assignments zählen diejenigen, deren Token im Browser gespeichert ist.
Dies umfasst einerseits die eigens erstellten Assignments, andererseits diejenigen, deren Lösungsliste bereits geöffnet und mit dem Token freigeschaltet wurde.
Diese Liste kann nützlich sein, wenn ein Link nicht mehr auffindbar ist.
Ein Klick auf einen Eintrag der Liste leitet auf die Liste mit Lösungen für das jeweilige Assignment weiter.

\begin{figure}
    \centering
    \includegraphics[width=\textwidth]{chapter/fulib.org/img/my-assignments.png}
    \caption{Liste der eigenen Assignments}
    \label{fig:my-assignments}
\end{figure}

\subsection{Lösen}\label{subsec:solution}

Studierende können ein Assignment lösen, indem sie dessen Link besuchen.
Daraufhin wird die in Abbildung~\ref{fig:assignment-solve} dargestellt Seite angezeigt.
Diese enthält alle beim Erstellen angegebenen Eckdaten, sowie die Beschreibung und Task-Liste des Assignments.
Darunter befindet sich der Editor für die Lösung.
In Abbildung~\ref{fig:assignment-solve} enthält dieser bereits die vom Ersteller vorgegebene Lösungsvorlage als Hilfe zum Einstieg.

\begin{figure}
    \centering
    \includegraphics[width=\textwidth]{chapter/fulib.org/img/assignment-solve.png}
    \caption{Lösungsseite für ein Assignment}
    \label{fig:assignment-solve}
\end{figure}

Sobald man das Scenario im Editor bearbeitet, beginnt die automatische Prüfung anhand der Tasks.
Dabei werden erfüllte Tasks grün und nicht erfüllte Tasks rot markiert.
Auch die erreichte Punktzahl wird ermittelt und angezeigt.
Bei nicht erfüllten Tasks lässt sich mit ``View Output'' die Ausgabe des Scenario-Compilers anzeigen, welche auf die Fehlerursache hinweist.
Abbildung~\ref{fig:solve-tasks} zeigt, wie dies bei einer Teillösung aussehen kann.
Der Studierende kann dann seine Lösung anpassen, um alle Tasks zu erfüllen.
Es ist jedoch auch möglich, eine teilweise oder vollständig fehlerhafte Lösung abzugeben.

\begin{figure}
    \centering
    \includegraphics[width=\textwidth]{chapter/fulib.org/img/solve-tasks.png}
    \caption{Erfüllte und unerfüllte Tasks}
    \label{fig:solve-tasks}
\end{figure}

Um die Lösung abzugeben, muss der Studierende zunächst Namen, Matrikelnummer und E-Mail-Adresse angeben.
Daraufhin kann mit ``Submit'' die Lösung eingereicht werden.
Dies öffnet das in Abbildung~\ref{fig:solution-submitted} dargestellte Fenster.

\begin{figure}
    \centering
    \includegraphics[width=\textwidth]{chapter/fulib.org/img/solution-submitted.png}
    \caption{Fenster nach Einreichen einer Lösung}
    \label{fig:solution-submitted}
\end{figure}

Auch für Lösungen wird ein Link und ein Token generiert.
Der Link dient dazu, die Lösung sowie deren Bewertung später einsehen zu können.
Dies ist aus Sicherheitsgründen nur mit dem Token möglich, welches zwar im Browser gespeichert wird, jedoch beim Öffnen der Lösungsseite mit einem anderen Gerät eingegeben werden muss.

Nach der Abgabe kann der Lösungsentwurf weiter bearbeitet werden.
Dazu muss lediglich das Bestätigungsfenster geschlossen werden.
Durch erneutes Betätigen von ``Submit'' wird die neue Lösung eingereicht;
die alte Abgabe bleibt jedoch unverändert und weiterhin einsehbar.
Auch bei einem späteren Besuch der Seite bleiben sämtliche Eingaben bestehen, da sie im Browser gespeichert werden.
Dadurch kann an einer angefangen Lösung bedenkenlos später weitergearbeitet werden, ohne sie zwischenzeitlich einreichen zu müssen.

Studierende können ihre abgegebenen Lösungen einsehen, indem sie die Seite unter Assignments > My Solutions im Footer öffnen.
Dort finden sie sämtliche eingereichte Versionen nach Assignment gruppiert.
Abbildung~\ref{fig:my-solutions} zeigt, wie dies aussehen kann.
Der Button ``Edit'' erlaubt die Bearbeitung und erneute Abgabe einer Lösung.
Genauer öffnet er nur die Bearbeitungsseite des Assignments, wo die zuletzt im Browser gespeicherte Lösung angezeigt wird.
Diese kann u.U.\ von der zuletzt abgegebenen Version abweichen, wenn Änderungen vorgenommen aber nicht eingereicht wurden.

\begin{figure}
    \centering
    \includegraphics[width=\textwidth]{chapter/fulib.org/img/my-solutions.png}
    \caption{Liste der abgegebenen Lösungen}
    \label{fig:my-solutions}
\end{figure}

\subsection{Korrigieren}\label{subsec:correcting}

Einstiegspunkt für die Korrektur ist die Seite, auf der abgegebene Lösungen für ein Assignment einsehbar sind.
Diese ist unter dem zweiten Link erreichbar, der nach Anlegen eines Assignments erzeugt wurde.
Der Kursleiter kann diesen an die Korrekteure zusammen mit dem Token weiterleiten.
Beim ersten Besuch des Links muss letzteres angebeben werden, um die Lösungen anzuzeigen.
Daraufhin zeigt sich die Tabelle, die in Abbildung~\ref{fig:solution-table} zu sehen ist.

\begin{figure}
    \centering
    \includegraphics[width=\textwidth]{chapter/fulib.org/img/solution-table.png}
    \caption{Tabelle der eingereichten Lösungen für ein Assignment}
    \label{fig:solution-table}
\end{figure}

Über der Tabelle befindet sich eine Übersicht der Aufgabenstellung inklusive Eckdaten, Beschreibung, Tasks und eventueller Musterlösung.
Dies vereinfacht die Korrektur, da keine separate Seite geöffnet werden muss, um die Anforderungen einzusehen.
Das Suchfeld erlaubt es, die angezeigten Lösungen einzuschränken.
Dies ist sowohl als Freitextsuche über alle Spalten als auch für bestimmte Spalten möglich.
So lässt sich beispielsweise mit \code{assignee:Peter} nach allen Lösungen suchen, die dem Korrekteur \code{Peter} zugewiesen sind.
Die Zuweisung erfolgt durch Ändern des Eingabefelds unter ``Assignee'' und wird sofort übernommen.
I.d.R.\ kann der Kursleiter diese Zuordnung durchführen, woraufhin die Korrekteure nach ihnen zugewiesenen Lösungen suchen können.

Die Tabelle gibt zunächst eine Übersicht über die Randangaben der Lösung.
Dazu gehören neben den Angaben zum Studierenden auch das Abgabedatum sowie die erreichte Punktzahl.
Nach der Deadline abgegebene Lösungen werden rot hervorgehoben.
Durch Klicken auf ``View'' gelangt man zur Detailseite der Lösung, die in Abbildung~\ref{fig:solution} dargestellt ist.

\begin{figure}
    \centering
    \includegraphics[width=\textwidth]{chapter/fulib.org/img/solution.png}
    \caption{Ansicht einer abgegebenen Lösung}
    \label{fig:solution}
\end{figure}

Hier wird erneut die Aufgabenstellung angezeigt.
Die Task-Liste zeigt die erreichte Punktzahl, welche mittels ``Grade'' durch den Korrekteur angepasst werden kann.
Dabei muss die zu vergebende Punktzahl, ein kurzer Kommentar und der eigene Name angegeben werden.
In der Historie werden alle Änderungen der Bewertung aufgelistet.
Auch Studierende können diese einsehen, jedoch nicht verändern, da dafür das Token des Assignments benötigt wird.
Abbildung~\ref{fig:grade-popover} zeigt das Fenster der Bewertung eines Tasks.
Aufgrund der Vergabe von Teilpunkten wurde der entsprechende Task gelb markiert, was bei der automatischen Bewertung nicht möglich ist.

\begin{figure}
    \centering
    \includegraphics[width=0.5\textwidth]{chapter/fulib.org/img/grade-popover.png}
    \caption{Fenster zur Bewertung eines Tasks}
    \label{fig:grade-popover}
\end{figure}

Am unteren Ende der Seite befindet sich der Kommentarbereich.
Hier können Studierende und Korrekteure Fragen, Antworten und Anmerkungen zur Bewertung hinterlassen.
Ist das Token des Assignments vorhanden und ein Kommentar wird erstellt, wird dieser mit einem Häkchen hinter dem Namen hervorgehoben.
Dies verhindert, dass Studierende den Kursleiter oder einen Korrekteur imitieren können, indem sie deren Namen angeben.

\subsection{Kurse}\label{subsec:courses}

Kurse sind Sammlungen von Assignments, die eine feste Reihenfolge vorgeben.
Sie ermöglichen die Vergabe von mehreren Aufgabenstellungen, die inhaltlich zusammengehören und aufeinander aufbauen können.
So ist es beispielsweise möglich, einen unbeaufsichtigten, von einer Vorlesung unabhängigen Kurs zu erstellen, mit dem die Scenario-Sprache erlernt werden kann.
Interessierte können diesen aufrufen und selbstständig Assignments bearbeiten, die nicht von Korrekteuren bewertet werden.

Kurse werden erstellt, indem im Footer Assignments > Create Course ausgewählt wird.
Dies öffnet die in Abbildung~\ref{fig:create-course} gezeigte Seite.

\begin{figure}
    \centering
    \includegraphics[width=\textwidth]{chapter/fulib.org/img/create-course.png}
    \caption{Formular zum Erstellen eines Kurses}
    \label{fig:create-course}
\end{figure}

Ein Kurs benötigt neben den zugehörigen Assignments einen Titel und eine Beschreibung.
Assignments müssen im Vorhinein mit dem in Unterabschnitt~\ref{subsec:creation} erläuterten Formular erstellt werden.
Diese lassen sich dann mit dem dafür vorgesehenen Feld dem Kurs hinzufügen.
Alternativ können beliebige Assignments von anderen Erstellern eingebunden werden, solange der zugehörige Link bekannt ist.
Die dabei entstehenden Liste erlaubt das Entfernen der hinzugefügten Assignments sowie das Ändern ihrer Reihenfolge.
Dafür werden die gleichen Steuerelemente wie in der Task-Liste beim Anlegen eines Assignments verwendet.
Zum Erstellen genügt nach Ausfüllen des Formulars ein Klick auf den Button ``Submit''.
Dabei wird ein Link generiert, unter dem der Kurs abrufbar ist.
Abbildung~\ref{fig:course-view} zeigt die Seite, die damit aufgerufen wird.

\begin{figure}
    \centering
    \includegraphics[width=\textwidth]{chapter/fulib.org/img/course-view.png}
    \caption{Kurs-Ansicht}
    \label{fig:course-view}
\end{figure}

Die Bearbeitung eines Assignments in einem Kurs unterscheidet sich nur geringfügig von der eines alleinstehenden Assignments.
Am linken Bildschirmrand kommt eine Seitenleiste hinzu, die Titel, Beschreibung und Verlauf des Kurses anzeigt.
Darin werden das aktuell angezeigte sowie bereits bearbeitete Assignments hervorgehoben.
Die Assignments eines Kurses können in beliebiger Reihenfolge bearbeitet werden.
Grund dafür ist, dass sie auch außerhalb des Kurses beliebig abruf- und bearbeitbar sind und somit die Einschränkung leicht umgangen werden könnte.
I.d.R.\ ist es jedoch sinnvoll, sich an den vorgegebenen Ablauf zu halten.
Bearbeitet man ein Assignment als Teil eines Kurses und reicht eine Lösung ein, verändert sich das Bestätigungsfenster der Abgabe.
In diesem kommt ein Button hinzu, der das Fortschreiten zum nächsten Assignment im Kurs erlaubt.

\subsection{Pattern Matching in Assignments}\label{subsec:assignment-pattern-matching}

Die vorherigen Beispiele für Assignments verwendeten nur einfache Funktionalität der Scenario-Sprache zum Prüfen von Bedingungen.
Es wird nun anhand der in Kapitel~\ref{ch:pattern-matching} vorgestellten Konzepte ein vollständiges Beispiel vorgestellt, dessen Aufgabe es sein soll, das im gleichen Kapitel erklärte Spiel zu modellieren.
Die Aufgabenstellung soll bewusst Freiheiten bei der Benennung von Klassen und Attributen erlauben;
lediglich die Struktur des zu erstellenden Modells wird vorgegeben.
Vorgaben werden durch die Tasks des Assignments gestellt, die nach Möglichkeit unabhängig voneinander sein sollen.
Dadurch soll es möglich sein, Teilpunkte zu erhalten.

Zu einem vollständigen Assignment gehören zunächst die Eckdaten.
Der Titel des Assignments kann beispielsweise ``Game Model'' lauten.
Name und E-Mail sind in diesem Fall nicht relevant, ebenso wenig die Deadline.
Als Beschreibung wäre etwa folgender Text vorstellbar:

\begin{quote}
    This assignment is about modelling a simple game with multiple players.
    Each player has one or more houses, each of which has a number of units.
    The game has a score that must be beaten by at least one house.
    The player who owns that house is then declared the winner.

    Model the objects and relationships described below.
\end{quote}

Dieser Beispieltext ist bewusst kurz gehalten, da die eigentlichen Aufgaben in den Task-Beschreibungen zu finden sind.
Nachfolgend sind alle Tasks mit dem zugehörigen Verifizierungscode sowie einer möglichen Punktzahl aufgelistet.

\begin{itemize}
    \item \textbf{Create a game object and give it a minimum score of 50.} (5 Punkte)

    \begin{mdcodeblock}
        We match some object g where some attribute is 50.
    \end{mdcodeblock}

    \item \textbf{Create three players: Alice, Bob and Charlie.
    Set their name in an attribute.} (15 Punkte)

    \begin{mdcodeblock}
        We match:
        - some object alice1 where some attribute matches '(?i)alice'
        - some object bob1 where some attribute matches '(?i)bob'
        - some object charlie1 where some attribute matches '(?i)charlie'.
    \end{mdcodeblock}

    \item \textbf{Link the players with the game.} (5 Punkte)

    \begin{mdcodeblock}
        We match:
        - some object alice1 where some attribute matches '(?i)alice' and with some link to g
        - some object bob1 where some attribute matches '(?i)bob' and with some link to g
        - some object charlie1 where some attribute matches '(?i)charlie' and with some link to g
        - some object g where some attribute is 50.
    \end{mdcodeblock}

    \item \textbf{Create four houses, with 30, 20, 40 and 60 units.} (10 Punkte)

    \begin{mdcodeblock}
        We match:
        - some object h1 where some attribute is 30
        - some object h2 where some attribute is 20
        - some object h3 where some attribute is 40
        - some object h4 where some attribute is 60.
    \end{mdcodeblock}

    \item \textbf{Link the first two houses with Alice, and the other two with Bob and Charlie.} (5 Punkte)

    \begin{mdcodeblock}
        We match:
        - some object h1 where some attribute is 30 and with some link to alice1
        - some object h2 where some attribute is 20 and with some link to alice1
        - some object h3 where some attribute is 40 and with some link to bob1
        - some object h4 where some attribute is 60 and with some link to charlie1
        - some object alice1 where some attribute matches '(?i)alice'
        - some object bob1 where some attribute matches '(?i)bob'
        - some object charlie1 where some attribute matches '(?i)charlie'.
    \end{mdcodeblock}
\end{itemize}

Eine Musterlösung könnte das folgende Scenario sein:

\begin{mdcodeblock}
    There is a Game with min-score 50.

    There is a Player with name Alice.
    There is a Player with name Bob.
    There is a Player with name Charlie.

    The game has players and is game of Alice, Bob and Charlie.

    There is a House with id a1 and with 30 units.
    There is a House with id a2 and with 20 units.
    There is a House with id b1 and with 40 units.
    There is a House with id c1 and with 60 units.

    Alice has houses and is player of a1 and a2.
    Bob has houses b1.
    Charlie has houses c1.
\end{mdcodeblock}

Dabei können die Klassennamen \code{Game}, \code{Player} und \code{House} sowie die Attribut-\ und Assoziationsnamen \code{min-score}, \code{name}, \code{players/game}, \code{units} und \code{houses/player} beliebig verändert werden, ohne dass die Lösung ihre Gültigkeit verliert.
Beim Schreiben der Lösung fällt auf, dass mit jedem Absatz ein weiterer Task erfüllt und damit grün markiert wird.
Lässt man den dritten oder den letzten Absatz aus, sind die Tasks 1, 2 und 4 noch immer erfüllt.
Dies wird durch die unabhängige Verifizierung erreicht.
So können Studierende, die beispielsweise nicht wissen, wie Assoziationen definiert werden, noch Teilpunkte auf das Aufgabenblatt erhalten, indem sie nur Objekte und Attribute anlegen.
Eine Teillösung, die diese beiden Aspekte einbezieht, könnte folgende sein:

\begin{mdcodeblock}
    There is the SimpleGame sg with score 50.

    There is the SimplePlayer p1 with name Alice.
    There is the SimplePlayer p2 with name Bob.
    There is the SimplePlayer p3 with name Charlie.

    There is the PlayerHouse p1h1 with 30 fighters.
    There is the PlayerHouse p1h2 with 20 fighters.
    There is the PlayerHouse p2h1 with 40 fighters.
    There is the PlayerHouse p3h1 with 60 fighters.
\end{mdcodeblock}
