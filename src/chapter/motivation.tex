\chapter{Zielsetzung}\label{ch:goals}

Auf hoher Ebene ist das Ziel dieser Arbeit, die Grundlage für eine E-Learning-Platform zu schaffen.
Diese soll zunächst auf das Thema der Datenmodellierung beschränkt sein, aber konzeptuell die Erweiterung auf andere Themengebiete auch außerhalb der Informatik erlauben.
Es soll möglich sein, Aufgaben bei dieser Platform einzureichen, die von Studierenden oder anderen Interessengruppen bearbeitet werden können.
Der Ersteller der Aufgabenstellung soll in der Lage sein, deren Anforderungen so zu spezifizieren, dass Lösungen automatisch bewertet werden können.
Die automatische Bewertung soll nicht nur nach Abgabe sondern auch während der Bearbeitung durchgeführt werden, um sofortiges Feedback zu ermöglichen.
Nach dem Einreichen von Lösungen sollen diese für Ersteller der Aufgabenstellung und erwählte weitere Personen zugänglich sein, um die automatische Bewertung bei Bedarf durch eine händische zu ergänzen.
Dies ist explizit optional, um die Bereitstellung von unbeaufsichtigten, öffentlich zugänglichen Aufgaben in Form von interaktive Tutorials zu ermöglichen.
Eine Erweiterung des Aufgabenkonzepts sieht vor, diese in Gruppen oder ganze Kurse zusammenfassen zu können.
Deren Ziel ist es, das Konzept auf größere Themengebiete übersichtlich auszuweiten.

Speziell im Kontext der Modellierung gilt es, gewisse Freiheiten in Lösungen von Aufgaben zu erlauben.
Dazu gehört besonders die Freiheit der Namensgebung, sofern diese nicht explizit vorgegeben ist.
Bei konkreten Modellierungsaufgaben soll jedoch auch die Verwendung von abweichenden Werten gestattet sein.
Der Aufgabenersteller soll in der Lage sein, nach eigenem Ermessen Anforderungen und Toleranzbereiche festzulegen.
Dies soll seitens des Lernenden den Fokus auf konzeptuelle Hintergründe der Modellierung ermöglichen, indem beispielspezifische Details weniger Bedeutung erhalten.
