\chapter{Zielsetzung}\label{ch:goals}

In erster Instanz gilt es, eine Sprache für textuelle Beispielszenarien zu formalisieren.
Sie soll aus englischen Sätzen und Ausdrücken bestehen, aber gleichzeitig Formatierung erlauben, um als Dokumentation einsetzbar zu sein.
Die Sprachfeatures sollen ausreichen, um Objekte und einfache Abläufe zu beschreiben.
Aus einem Beispielszenario sollen sich Objekt-\ und Klassendiagramme sowie Tests und Dokumentation ableiten lassen.
Es ist deshalb entscheidend, dass die Semantik eindeutig definiert ist.
Da eine natürliche Sprache wie Englisch nicht immer eindeutig ist, müssen die Texte eingeschränkt in Struktur und Wortwahl sein.
Deshalb ist es explizit nicht Teil der Zielsetzung, dass Personen ohne technischen Hintergrund die Beispielszenarien schreiben können;
dies soll weiterhin Aufgabe von Softwareentwicklern sein.
Das Leseverständnis soll jedoch für jeden gewährleistet sein.

Diese Arbeit widmet sich der Aufgabe, die Grundlage für eine Online-Lehrplattform zu erschaffen.
Diese soll zunächst auf das Thema der Datenmodellierung beschränkt sein, aber konzeptuell die Erweiterung auf andere Themengebiete auch außerhalb der Informatik erlauben.
Es soll möglich sein, Aufgaben bei dieser Plattform einzureichen, die von Studierenden oder anderen Interessengruppen bearbeitet werden können.
Der Ersteller der Aufgabenstellung soll in der Lage sein, deren Anforderungen so zu spezifizieren, dass Lösungen automatisch bewertet werden können.
Die automatische Bewertung soll nicht nur nach der Abgabe, sondern auch während der Bearbeitung durchgeführt werden, um sofortiges Feedback zu ermöglichen.
Nach dem Einreichen von Lösungen sollen sie für den Ersteller der Aufgabenstellung und ausgewählte weitere Personen zugänglich sein, um die automatische Bewertung bei Bedarf durch eine manuelle zu ergänzen.
Die manuelle Bewertung soll jedoch explizit optional sein, um die Bereitstellung von unbeaufsichtigten, öffentlich zugänglichen Aufgaben in Form von interaktiven Tutorials zu ermöglichen.
Eine Erweiterung des Aufgabenkonzepts sieht vor, diese in Gruppen oder ganze Kurse zusammenfassen zu können.
Deren Ziel ist es, das Konzept auf größere Themengebiete übersichtlich auszuweiten.

Speziell im Kontext der Datenmodellierung gilt es, gewisse Freiheiten in Lösungen von Aufgaben zu erlauben.
Dazu gehört besonders die Freiheit der Namensgebung, sofern diese nicht explizit vorgegeben ist.
Bei konkreten Modellierungsaufgaben soll jedoch auch die Verwendung von abweichenden Werten gestattet sein.
Der Aufgabenersteller soll Anforderungen und Toleranzbereiche nach eigenem Ermessen festlegen können.
Dies soll seitens des Lernenden den Fokus auf konzeptuelle Hintergründe der Modellierung ermöglichen, indem beispielspezifische Details weniger Bedeutung erhalten.
