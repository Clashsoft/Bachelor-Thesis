\chapter{Fulib Scenarios}\label{ch:fulib-scenarios}

Dieses Kapitel befasst sich mit dem natürlichsprachlichen Aspekt dieser Arbeit.
Es betrachtet das FulibScenarios-Projekt, das in Vorbereitung auf die weiteren Inhalte entstanden ist.
Dessen Ziel war es, eine Beschreibungssprache für Objektstrukturen und Abläufe zu entwickeln.
Diese Abläufe werden \emph{Scenarios} genannt und sind mit Stories aus dem Story Driven Modeling gleichzusetzen.
Sie ermöglichen dabei eine einfache Form der Dokumentation.
Die Verständlichkeit wird sowohl für Entwickler als auch für Personen mit nicht-technischem Hintergrund gewährleistet.
Entwickler können Scenarios unter Beaufsichtigung und Beratung mit Domänenexperten erstellen, um die Anforderungen von Software zu erarbeiten.
Dieser Vorgang wurde bereits in ``Explaining Business Process Software with Fulib-Scenarios''~\cite{explain} erforscht.

Scenarios haben als Werkzeug sowohl in der Modellierung als auch beim Testen eine Bedeutung.
Einerseits lassen sich aus diesen mit vergleichsweise geringem Aufwand umfangreiche Datenmodelle erzeugen, aus denen automatisch Code generiert werden kann.
Sie erfüllen damit eine ähnliche Funktion wie das Eclipse Modeling Framework~\cite{emf} (EMF) oder das SDMLib-Framework~\cite{networkparser}, das als Vorgänger von FulibScenarios gilt.
Der zugrundeliegende Ansatz beim Erstellen von Datenmodellen mit FulibScenarios unterscheidet sich jedoch stark von beiden Frameworks.
Während das EMF auf grafische Werkzeuge setzt, wird in SDMLib das Datenmodell mit imperativem Java-Code definiert.
FulibScenarios hingegen setzt auf eine textuelle Beschreibung konkreter Objekte, aus der das Datenmodell abgeleitet wird.
Vorteil davon ist, dass aufgrund des konkreten Ablaufs automatisch ein Test abgeleitet werden kann.
Dieser kann auch nach Änderungen an der Software sicherstellen, dass die ursprüngliche, vom Scenario beschriebene Story noch gültig und durchführbar ist.

Der folgende Abschnitt befasst sich zunächst mit den Grundlagen von Scenarios.
Daraufhin wird anhand eines Beispiels erklärt, wie ein Scenario verarbeitet wird, um daraus ein Datenmodell und Tests zu generieren.

\section{Sprache}\label{sec:language}

\section{Compiler}\label{sec:compiler}

Damit eine Programmiersprache verwendbar ist, muss sie nicht nur eine Spezifikation haben, sondern auch ausführbar sein.
Dies kann durch direktes Ausführen des Programmtexts geschehen;
in diesem Fall wird das ausführende Programm Interpreter genannt.
FulibScenarios verwendet jedoch den zweiten Ansatz der Kompilierung.
Dabei werden die vorliegenden Quelltexte zunächst in eine andere Text- oder Binärsprache übersetzt.
Ziel der Kompilierung von FulibScenarios sind Java-Quelltextdateien.
Die Übersetzung von Markdown-Scenario-Quelltexte in Java-Quelltexte wird durch den Scenario-Compiler durchgeführt.
Dessen Architektur und Implementierung sind Inhalt dieses Abschnitts.
Dabei wird des öfteren auf das Beispiel in Listing~\ref{lst:CompilationExample.md} Bezug genommen, dessen Übersetzung in den einzelnen Teilabschnitten stückweise erarbeitet wird:

% TODO markdown
\codelisting{text}{chapter/fulib-scenarios/scenarios}{CompilationExample.md}{Beispiel-Szenario für diesen Abschnitt}

\subsection{Architektur}\label{subsec:compiler-architecture}

\todo{
Visitor Pattern~\cite{gof-design-patterns}.
Dragon Book~\cite{dragonbook}.
}

\begin{figure}
	\includegraphics[width=\textwidth]{chapter/fulib-scenarios/img/architecture.pdf}
	\caption{Compiler-Architektur}
	\label{fig:compiler-architecture}
\end{figure}

\subsection{Frontend - ANTLR v4}\label{subsec:frontend-antlr4}

\todo{
Warum ANTLR4 sehr gut für die Sprachstruktur geeignet ist~\cite{adaptive-ll-star,antlr4-reference}.
}

\subsection{AST, Analyse und Transformation}\label{subsec:data-model-gentreesrc}

\todo{
AST mit GenTreeSrc;
eigenes Projekt, kurze Erklärung.
Prä-Transformation: Gruppieren von Sätzen zu Methoden
Analyse: Variablen, Konflikte bei Attributen und Assoziationen, Programmstruktur.
Transformation: Bauen von Klassenmodell, Vereinfachung des AST in Vorbereitung auf CodeGen.
}

\subsection{Codegenerierung - Fulib}\label{subsec:codegen-fulib}

\todo{
Modellgenerierung mit Fulib.
Testgenerierung selbst.
}

