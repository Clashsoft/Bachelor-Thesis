% geometry
\usepackage[bindingoffset=1cm, left=2.5cm, right=2.5cm, top=2.5cm, bottom=2.5cm]{geometry}

% Headline
\usepackage{fancyhdr}
\pagestyle{fancy}
\renewcommand{\chaptermark}[1]{\markboth{\thechapter\ #1}{}}
\lhead{\leftmark} \rhead{\thepage}
\cfoot{}
\fancypagestyle{plain}{}

% Select input encodung, usually utf8 is the best choice, on windows, \usepackage[latin1]{inputenc} maybe required
\usepackage[utf8]{inputenc}
\usepackage[T1]{fontenc}

% Colors
\usepackage{color}
\usepackage{colortbl}

% Tables
\usepackage{tabularx}
\usepackage{multirow}

% Drawing graphs etc.
\usepackage{pgf}
\usepackage{tikz}
\usetikzlibrary{arrows,automata}

% math
\usepackage{amsmath}

% lists
\usepackage{paralist}

% Figures
\usepackage{graphicx, wrapfig}

% Hyperlinks
\usepackage[hyphens]{url}
\usepackage{hyperref}
\hypersetup{colorlinks, citecolor=black, linkcolor=black, urlcolor=black}

% Minted
\usepackage[chapter]{minted}
%\usemintedstyle{xcode}
\setminted{frame=single,tabsize=2,linenos}

\newmintinline[code]{text}{breaklines}
\newmintinline[mdcode]{md}{breaklines}
\newmintinline[jcode]{java}{breaklines}

\newminted[codeblock]{text}{autogobble,frame=none,linenos=false,breaklines}
\newminted[mdcodeblock]{md}{autogobble,frame=none,linenos=false,breaklines}
\newminted[jcodeblock]{java}{autogobble,frame=none,linenos=false}

\newcommand{\codelisting}[4]{%
    \begin{listing}[htp]
        \inputminted{#1}{#2/#3}
        \vspace{-3ex}
        \caption{#4}
        \label{lst:#3}
    \end{listing}%
}

% list of abbreviations
\usepackage[printonlyused]{acronym}

% Set line pitch
\usepackage{setspace}
\onehalfspacing              % anderthalbzeilig (oder auch \doublespace)

% Newcommand TODO (red in text)
\newcommand{\todo}[1]{\textcolor{red}{TODO: #1}}

% Newcommand TODOM (red at border)
\newcommand{\todom}[1]{\marginpar{\parbox{1.5cm}{\textcolor{red}{TODO:\\ #1}}}}

%fancyBox
%\usepackage{fancybox}

% Layout corrections (Schusterjungen)
\clubpenalty = 10000
% Layout corrections (Hurenkinder) 
\widowpenalty = 10000
\displaywidowpenalty = 10000

% Figures
\usepackage{caption}
\usepackage[hypcap=true,labelformat=simple]{subcaption}
\renewcommand{\thesubfigure}{(\alph{subfigure})}

% Tables
\usepackage{booktabs}

% Frequently used column types
\newcolumntype{C}[1]{>{\centering\arraybackslash}p{#1}} % centering column type with fixed width
\newcolumntype{R}[1]{>{\raggedleft\arraybackslash}p{#1}} % right aligned column type with fixed width
\newcolumntype{L}[1]{>{\raggedright\arraybackslash}p{#1}} % left aligned column type with fixed width

% Shortcuts for referencing floats:
\newcommand{\fig}[1]{\figurename~\ref{#1}} %shortcut for a figure reference
\newcommand{\tab}[1]{Table~\ref{#1}} %shortcut for a table reference
\newcommand{\eq}[1]{(\ref{#1})} %shortcut for an equation reference
\newcommand{\lst}[1]{Listing~\ref{#1}} %shortcut for a listing reference
\newcommand{\sect}[1]{Section~\ref{#1}} %shortcut for a Section reference

